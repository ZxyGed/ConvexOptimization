\documentclass[11pt]{ctexart}         %编辑中文文档
\usepackage{amsmath,amsfonts,amssymb,nccmath} %数学工具包
\usepackage[margin=1in]{geometry}     %页面设置,边距,页幅大小等
\usepackage{enumerate}                %有序列表工具包
%\usepackage{hyperref}                 %方便设置超链接
\usepackage{fancyhdr}                 %设置页眉页脚
\usepackage{float}                    %设置图片环绕样式
\usepackage{graphicx}                 %插入图片
\usepackage{color}
\usepackage{xcolor}
\usepackage{adjustbox}
\usepackage{pifont}
\usepackage{extarrows}
\usepackage{enumitem}
\usepackage{setspace}
\usepackage{subfigure}
\usepackage{etoolbox}

\CTEXsetup[format={\Large\bfseries}]{section} %设置section标题左对齐,挺迷的
\usepackage[
	pdfstartview=FitH,
	CJKbookmarks=true,
	bookmarksnumbered=true,
	bookmarksopen=true,
	colorlinks,
	pdfborder=001,
	linkcolor=blue,
	anchorcolor=blue,
	citecolor=blue,
]{hyperref}
\hypersetup{hidelinks} %去除目录项以及超链接的方框

\pagestyle{fancy}                     %设置页眉页脚
\fancyhead{}                          %清除默认页眉
\fancyfoot{}                          %清除默认页脚
\fancyhead[L]{\slshape{Convex Optimization}}%设置自定义页眉左侧,slshape表斜体	
\fancyhead[R]{\slshape Chapter5 Duality}	
\fancyfoot[C]{\thepage}
\parindent 0ex %latex首段不缩进,其后段落缩进,\parindent为其后段落缩进的长度
\setlength{\parskip}{1em}  %段落间距
\renewcommand{\baselinestretch}{1.5}  %设置行距
\renewcommand{\vec}[1]{\boldsymbol{#1}}  %设置粗体向量
\newcommand{\tabincell}[2]{\begin{tabular}{@{}#1@{}}#2\end{tabular}}%表格内换行

\newenvironment{myenumerate}{\begin{enumerate}[itemsep=0pt,parsep=0pt,topsep=0pt]}{\end{enumerate}}
\newcommand{\liset}{itemsep=0pt,parsep=0pt,topsep=0pt}
\newcommand{\rebacklinespread}[1][-12pt]{\vspace{#1}}
\newcommand{\oneline}[1][12pt]{\vspace{#1}}
\newcommand{\premise}[1][dom\ f]{\forall\ x,y\in #1,\ \theta\in [0,1]}
\newcommand{\linearcombine}[2]{\theta #1+(1-\theta)#2}
\newcommand{\ii}{\,\in\,}
\newcommand{\dom}[1]{$dom\ #1$}
\newcommand{\rs}[2][R]{#1^{#2}} %Rspace or matrix space
\newcommand{\rl}[2][R]{#1_{#2}} %Rlimit or matrix space
\newcommand{\rls}[3][R]{#1_{#2}^{#3}} %R limit space or matrix limit space

\newcommand{\trans}[1]{#1^T#1} %transpose
\newcommand{\ftrans}[1]{\left(#1\right)^T#1} %fraction transpose
\newcommand{\mirror}[2]{#1#2#1}
\newcommand{\ba}[1]{\begin{align*}#1\end{align*}}
\newcommand{\bc}[1]{\begin{cases}#1\end{cases}}
\newcommand{\bcenter}[1]{\begin{center}#1\end{center}}
\newcommand{\bef}[2][0pt]{\begin{fleqn}[#1]#2\end{fleqn}}
\newcommand{\eq}[1]{\begin{equation}#1\end{equation}}
\newcommand{\li}[3][例]{
	#1:#2\\ 
	\phantom{#1:}\begin{minipage}[t]{0.9\linewidth}%注意这里phantom和minipage之间不能换行
	\setlength\parskip{12pt}
	\setlength{\belowdisplayskip}{0pt}%公式后的距离
	\setlength{\abovedisplayskip}{0pt}%公式前的距离
	\setlength{\belowdisplayshortskip}{0pt}%如果公式后的句子很短,将被采用[只对equation环境有效]
	\setlength{\abovedisplayshortskip}{0pt}%如果公式前的句子很短,将被采用	[只对equation环境有效]
	#3
	\end{minipage}
	\oneline}
\newcommand{\sune}{\parbox{1em}{$ \Rightarrow $\\ [-10pt] $ \nLeftarrow $}} % Sufficiently unnecessary
\newcommand{\usne}{\parbox{1em}{$ \nRightarrow $\\ [-10pt] $ \Leftarrow $}} % unsufficiently necessary
\newcommand{\paint}[2][red]{{\color{#1}#2}} %着色
\newcommand{\VV}[1]{\Vert #1 \Vert}

\begin{document}
% 必须在正文区设置
\setlength{\belowdisplayskip}{6pt}%公式后的距离
\setlength{\abovedisplayskip}{6pt}%公式前的距离
\setlength{\belowdisplayshortskip}{0pt}%如果公式后的句子很短,将被采用[只对equation环境有效]
\setlength{\abovedisplayshortskip}{0pt}%如果公式前的句子很短,将被采用	[只对equation环境有效]

\hrule height 4pt
\begin{Large}
	\textbf{Essence of the lecture (29)}\\
\end{Large}
\begin{large}
	\textbf{凸问题:} 
\end{large}
\vspace{-16pt}
\begin{itemize} \setlength{\itemsep}{0pt}
	% 使用\displaystyle来解决行内时bigcap变形
	\item lagrangian function, lagrange dual function
\end{itemize}
% 设置hrule的宽度为4pt
\hrule height 4pt

\textbf{Lagrangian function/Lagrangian: }$ L(x,\lambda,\upsilon)=f_0(x)+\sum_{i=1}^{m}\lambda_i f_i(x)+\sum_{i=1}^{p}\upsilon_ih_i(x) $

\textbf{Lagrange dual function/Dual function: }$ g(\lambda,\upsilon)=\inf\limits_{x\in D}L(x,\lambda,\upsilon) $

\textbf{两条结论:}\\
1)$ g(\lambda,\upsilon) $一定是凹函数,因为$ \sup_{x\in D}L(x,\lambda,\upsilon) $是凸函数(分段极大,保凸操作)\\
2)$ \forall \lambda\geq 0,\ \forall \upsilon,\ g(\lambda,\upsilon)\leq p^*$,其中$ p^* $为原问题的最优值\\
\phantom{2)}证明:设$ x^* $为原问题的最优解,则$ f_i(x^*)\leq 0,\ h_i(x^*)=0$\\
\phantom{2)}回代有$ L(x^*,\lambda,\upsilon)=f_0(x^*)+\sum_{i=1}^{m}\lambda_i f_i(x^*)+\sum_{i=1}^{p}\upsilon_ih_i(x^*)\leq f_0(x^*)=p^*$

\li{$\min\ c^Tx\quad s.t.\ Ax=b,\ x\geq0$}{
	首先列出其拉格朗日方程
	\begin{fleqn}
		\begin{align*}
			L(x,\lambda,\upsilon)&=c^Tx+\lambda^T(-x)+\upsilon^T(Ax-b)\\
			&=(c^T-\lambda^T+\upsilon^TA)x-\upsilon^Tb\\
			g(\lambda,\upsilon)&=\inf_{x\in D}L(x,\lambda,\upsilon)=
			\bc{-b^T\upsilon\quad &A^T\upsilon-\lambda+c=0\\-\infty\quad &otherwise}
		\end{align*}
	\end{fleqn}
}

\li{$\min\ X^TWX\quad s.t.\ x_i=\pm 1,\ i=1,\dots,m$ }{
	该例中目标函数不一定是凸函数,同时约束条件不是凸集,首先将约束项化为$ x_i^2=1 $
}


\end{document}