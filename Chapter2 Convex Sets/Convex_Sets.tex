\documentclass[11pt]{ctexart}         %编辑中文文档
\usepackage{amsmath,amsfonts,amssymb} %数学工具包
\usepackage[margin=1in]{geometry}     %页面设置,边距,页幅大小等
\usepackage{enumerate}                %有序列表工具包
%\usepackage{hyperref}                 %方便设置超链接
\usepackage{fancyhdr}                 %设置页眉页脚
\usepackage{float}                    %设置图片环绕样式
\usepackage{graphicx}                 %插入图片
\usepackage{color}
\usepackage{xcolor}
\usepackage{adjustbox}


\CTEXsetup[format={\Large\bfseries}]{section} %设置section标题左对齐,挺迷的
\usepackage[
	pdfstartview=FitH,
	CJKbookmarks=true,
	bookmarksnumbered=true,
	bookmarksopen=true,
	colorlinks,
	pdfborder=001,
	linkcolor=blue,
	anchorcolor=blue,
	citecolor=blue,
]{hyperref}
\hypersetup{hidelinks} %去除目录项以及超链接的方框

\pagestyle{fancy}                     %设置页眉页脚
\fancyhead{}                          %清除默认页眉
\fancyfoot{}                          %清除默认页脚
\fancyhead[L]{\slshape{Convex Optimization}}%设置自定义页眉左侧,slshape表斜体	
\fancyhead[R]{\slshape Chapter2 Convex Sets}	
\fancyfoot[C]{\thepage}
\parindent 0ex %latex首段不缩进,其后段落缩进,\parindent为其后段落缩进的长度
\setlength{\parskip}{1em}  %段落间距
\renewcommand{\baselinestretch}{1.5}  %设置行距
\renewcommand{\vec}[1]{\boldsymbol{#1}}  %设置粗体向量
\newcommand{\tabincell}[2]{\begin{tabular}{@{}#1@{}}#2\end{tabular}}%表格内换行

\begin{document}
\hrule height 4pt
\begin{Large}
	\textbf{Essence of the lecture (3/4)}
\end{Large}
\begin{center}
\begin{tabular}{|c|c|c|}
	\hline
	仿射集 & 凸集 & 凸锥 \\
	\hline
	%需要将公式内换行写为两条公式,否则tabincell会报错(但是效果是对的)
	\tabincell{c}{$\theta_1+\dots+\theta_k=1$\\ $\theta_1,\dots,\theta_k\in R$} & 
	\tabincell{c}{$\theta_1+\dots+\theta_k=1$\\ $\theta_1,\dots,\theta_k\in [0,1]$} & 
	$\theta_1,\dots,\theta_k \geq 0$\\
	\hline
\end{tabular}
\begin{align*}
	%单数列左对齐,奇数列右对齐,第二列空掉,使得两列都为左对齐
	&\text{直线:}&&\text{线段:}\\
	&x_1\neq x_2 \in R^n,\ \theta\in R  &&\theta\in R,\ \theta\in[0,1] \\ 
	&y=\theta x_1+(1-\theta)x_2  &&y=\theta x_1+(1-\theta)x_2 
\end{align*}
\end{center}
% 设置hrule的宽度为4pt
\hrule height 4pt

\textbf{仿射集(Affine Sets):}若$\forall x_1,x_2\in C$,则连接$x_1,x_2$的{\color{red}\textbf{直线}}也在集合内\\
\phantom{仿射集(Affine Sets):}$\Rightarrow$因此直线是仿射集,而线段不是

\textbf{仿射组合:}设$x_1,\dots,x_k\in C,\ \theta_1,\dots,\theta_k\in R,\theta_1+\dots+\theta_k=1,\ \theta_1x_1+\dots+\theta_kx_k$称为仿射组合\\
\phantom{仿射组合:}$\Rightarrow$若C为仿射集,则仿射组合也在C内,证明如下
\begin{align*}
	& \frac{\theta_1}{\theta_1+\theta_2}x_1+\frac{\theta_2}{\theta_1+\theta_2}x_2\in C\\
	& (\theta_1+\theta_2)\left(\frac{\theta_1}{\theta_1+\theta_2}x_1+\frac{\theta_2}{\theta_1+\theta_2}x_2\right)+(1-\theta_1-\theta_2)x_3\in C\\
	&\Leftrightarrow\theta_1 x_1+\theta_2 x_2+\theta_3 x_3 \in C,\quad \theta_1+\theta_2+\theta_3=1
\end{align*}
\textbf{仿射包:}对任意集合C,仿射包$(aff\ C)$是包含C的最小仿射集,
$$aff\ C=\{\theta_1x_1+\dots+\theta_kx_k|\forall x_1,\dots,x_k\in C,\theta_1+\dots+\theta_k=1\}$$
\hrulefill%注意使用的是hrulefill,在tabular外不能使用hline,否则会报错

$V=C-x_0=\{x-x_0|x\in C\}\ \forall x_0\in C\Rightarrow$\textbf{与C相关的子空间},即将C平移经过原点\\
求证:$v_1,\ v_2\in V,\forall \alpha,\beta\in R\quad \alpha v_1+\beta v_2\in V$\\
即证:$\alpha v_1+\beta v_2+x_0\in C\Longleftrightarrow \alpha (v_1+x_0)+\beta (v_2+x_0)+(1-\alpha-\beta)x_0\in C$\\ 
\phantom{即证:}由于$v_1+x_0,\ v_2+x_0,\ x_0\in C$,得证

线性方程组的解集是仿射集\\
证明:$C=\{x|Ax=b\},\ A\in R^{m\times n},\ b\in R^m,\ x\in R^n$\\
\phantom{证明:}由$Ax_1=b,\ Ax_2=b$有$\theta x_1+(1-\theta)x_2\in C$

\hrulefill

\textbf{凸集:}一个集合为凸集,当且仅当任意两点的{\color{red}\textbf{线段}}在C内\\
\phantom{凸集:}即$\forall x_1,x_2\in C,\ \forall \theta\in [0,1],\ \theta x_1+(1-\theta)x_2\in C$

\textbf{凸组合:}$x_1,\dots,x_k\in C,\ \theta_1,\dots,\theta_k\in [0,1],\ \theta_1+\dots+\theta_k=1,\ \theta_1x_1+\dots+\theta_kx_k$\\
\phantom{凸组合:}凸集包含其任意元素的凸组合

\textbf{凸包:} $Conv(C)=\{\theta_1x_1+\dots+\theta_kx_k|x_1,\dots,x_k\in C,、 \theta_1,\dots,\theta_k\in [0,1],\ \theta_1+\dots+\theta_k=1\}$\\
\phantom{凸包:}即对任意集合C,包含C的最小凸集

\textbf{锥:}$\forall x\in C,\ \theta\geq 0\Rightarrow \theta x\in C$

\textbf{凸锥:}$\forall x_1,x_2\in C,\ \theta_1,\theta_2\geq 0\Rightarrow \theta_1x_1+\theta_2x_2\in C$

\textbf{凸锥组合:}$\theta_1x_1+\dots+\theta_kx_k,\ \theta_1,\dots,\theta_k\geq 0$

\textbf{凸锥包:}$\{\theta_1x_1+\dots+\theta_kx_k|x_1,\dots,x_k\in C, \theta_1,\dots,\theta_k\geq 0\}$

\vspace{16pt}

\adjustbox{minipage=\textwidth,cfbox=red}{仿射集一定是凸集,凸锥一定是凸集;\\空集既是仿射集,又是凸集,又是凸锥;\\
	只有一个元素的集合一定是仿射集和凸集,是否是凸集,取决于该元素是否为原点(因为凸锥一定过原点)}


\newpage
\hrule height 4pt
\begin{Large}
	\textbf{Essence of the lecture (5/6)}\\
\end{Large}
\begin{large}
	\textbf{几种重要的凸集:} 
\end{large}
\vspace{-16pt}
\begin{itemize} \setlength{\itemsep}{0pt}
	\item $R^n$空间,$R^n$空间的子空间
	\item 任意直线(若过原点也为凸锥),任意线段,射线$\{x_0+\theta v|\theta\geq 0,\ x\in R^n,\ \theta\in R,\ v\in R^n\}$ 
	\item 超平面与半空间
	\item 球和椭球
	\item 多面体(Polyhedron)和单纯形(Simplex)
	\item 对称矩阵集合,对称半正定矩阵集合,对称正定矩阵集合
\end{itemize}
\hrule height 4pt

\textbf{超平面:}$\{x\mid a^Tx=b\}$,凸集,仿射集,是否过原点(凸锥)

\textbf{半空间:}超平面的衍生概念,$\{x\mid a^T\geq b\},\ \{x\mid a^Tx\leq b\}$,凸集,非仿射集,是否过原点(凸锥)

\textbf{球:}$B(x_c,r)=\{x\mid \Vert x-x_c\Vert_2\leq r\}=\{x\mid \sqrt{(x-x_c)^T(x-x_c)}\leq r\}$为凸集\\
证明:$\forall \theta\in[0,1]$,取$f(x)=\Vert x-x_c\Vert_2$\\
\phantom{证明:}$\Vert \theta x_1+(1-\theta)x_2-x_c\Vert_2=\Vert \theta (x_1-x_c)+(1-\theta)(x_2-x_c)\Vert_2$\\
\phantom{证明:}$\leq\ \theta\Vert x_1-x_c\Vert_2+(1-\theta)\Vert x_2-x_c\Vert_2$\\
\phantom{证明:}\textcolor{red}{这里主要利用了三角不等式(范数的条件2)},$\Vert a\Vert+\Vert b\Vert\geq \Vert a+b\Vert$

\textbf{椭球:}$\varepsilon(x_c,P)=\{x\mid (x-x_c)^TP^{-1}(x-x_c)\leq 1\},\ x_c\in R^n,\ P\in S_{++}^n$\\
\phantom{椭球:}其中P为对角矩阵,对角线上为矩阵的奇异值的平方,矩阵的奇异值对应了椭球的半轴长\\
\phantom{椭球:}矩阵A的奇异值为$\sqrt{eig(A^TA)}$,需要注意方针的特征值和奇异值也可能不等

\textbf{多面体(polyhedron):}$\{x\mid a_j^Tx\leq b_j,\ j=1,\dots,m,\ c_j^T=d_j,\ j=1,\dots,p\}$\\
可以看作半空间和超平面的交集,所以是凸集

\textbf{单纯形:}$R^n$中选择$v_0,\dots,v_k$共$k+1$个点,使$v_1-v_0,\dots,v_k-v_0$线性无关,则与上述点相关的单纯形为$C=Conv\{v_0,\dots,v_k\}=\{\theta_0v_0+\dots+\theta_kv_k\mid \theta\geq 0,1^T\theta =1\}$,即找到这k个点的凸包

注意$\{x\mid x\leq 0\}$是凸集/多面体/单纯形,即一维空间下取$x_1=0,\ x_2=-\infty$的凸包

\pagebreak
求证:单纯形是多面体的一种\\
证明:定义$y=[\theta_1,\dots,\theta_k],\ y\geq 0,\ 1^Ty\leq 1$(注意舍弃了$\theta_0$),$B=[v_1-v_0,\dots,v_k-v_0]\in R^{n\times k}$\\
\phantom{证明:}$x\in C\Leftarrow x=\theta_0v_0+\dots+\theta_kv_k=v_0+\theta_1(v_1-v_0)+\dots+\theta_k(v_k-v_0)=v_0+By$\\[8pt]
\phantom{证明:}B满秩,$rank(B)=k\Rightarrow$通过非奇异(可逆)矩阵$A=\left[\begin{array}{c}A_1\\A_2\end{array}\right]\in R^{n\times n}$,$\left[\begin{array}{c}A_1\\A_2\end{array}\right]B=\left[\begin{array}{c}I_k\\0\end{array}\right]$\\
\phantom{证明:}$Ax=Av_0+ABy\Rightarrow\left[\begin{array}{c}A_1\\A_2\end{array}\right]x=\left[\begin{array}{c}A_1\\A_2\end{array}\right]v_0+\left[\begin{array}{c}I_k\\0\end{array}\right]y$\\[8pt]
\phantom{证明:}利用$y\geq 0,\ 1^Ty\leq 1$,有$A_1x\geq A_1v_0,\ 1^TA_1x\leq 1^TAv_0+1$\\
\phantom{证明:}则单纯形中的x可以表示为$\{x\mid A_1x\geq A_1v_0,\ 1^TA_1x\leq 1^TAv_0+1,A_2x=A_2v_0\}$

\textbf{对称矩阵集合:}$S^n=\{x\in R^{n\times n}\mid X=X^T\}$

\textbf{对称半正定矩阵集合:}$S^n_{+}=\{x\in R^{n\times n}\mid X=X^T,\ X\succeq 0\}$

\textbf{对称正定矩阵集合:}$S^n_{++}=\{x\in R^{n\times n}\mid X=X^T,\ X\succ 0\}$

求证:$S_+^n$是凸锥
证明:$\forall \theta_0,\theta_1\geq 0,\ \forall A,B\in S_+^n$,对于$\theta_1A+\theta_2B$,首先对称性显然成立\\
\phantom{证明:}$\forall x\in R^n,\ X^TAX\geq 0,\ X^TBX\geq 0,\ \theta_1X^TAX+\theta_2X^TBX\geq 0$,$\theta_1A+\theta_2B\succ 0$,得证

求证:$S_{++}^n$不是凸锥\\
证明:首先考虑n=1的情况,不过原点\\
\phantom{证明:}高维情况下$\theta_1X^TAX+\theta_2X^TBX\geq 0\notin S^n_{++}$,因为$\theta_1,\theta_2$可同时为0

\newpage
\hrule height 4pt
\begin{Large}
	\textbf{Essence of the lecture (7/8)}\\
\end{Large}
\begin{large}
	\textbf{几种重要的保凸运算:} 
\end{large}
\vspace{-16pt}
\begin{itemize} \setlength{\itemsep}{0pt}
	% 使用\displaystyle来解决行内时bigcap变形
	\item 凸集的交操作
	\item 仿射变换(缩放与位移,线性矩阵不等式)
	\item 透视函数
	\item 线性分数函数(仿射和透视的融合)
\end{itemize}
\hrule height 4pt

\textbf{交集:}若$S_0$为凸集,$\forall a\in A$,则$\displaystyle\bigcap_{a\in A}S_a$为凸集

\textbf{仿射函数:}$f:R^n\to R^m$是仿射的({\color{red}线性映射}),当{\color{red}$f=Ax+b$}$,\ A\in R^{m\times n},\ b\in R^m$\\
\phantom{仿射函数:}若$S\in R^n$为凸,则$f(S)=\{f(x)\mid x\in S\}$为凸

\textbf{缩放与位移:}$\alpha S=\{\alpha x\mid x\in S\}\qquad S+a=\{x+a\mid x\in S\}$

求证:两个凸集的和仍旧是凸集,定义凸集的和为($S_1+S_2=\{x+y\mid x\in S_1,\ y\in S_2\}$)\\
证明:因为仿射映射是从一个集合的角度应用的,所以要将两个集合融合为一个\\
\phantom{证明:}定义$S_1\times S_2=\{(x,y)\mid x\in S_1,\ y\in S_2\}$,由凸集定义显然这是一个凸集\\
\phantom{证明:}则令$f(x,y)=x+y$,有$S_1+S_2=f(S_1\times S_2)$,由仿射变换保凸知,凸集和仍为凸集

\textbf{线性矩阵不等性(LMI):}$A(x)=x_1A_1+\dots+x_nA_n\preceq B,\ B,\ A_i,\ x_i\in S^m$\\
求证:LMI的解为凸集,即$\{x\mid A(x)\preceq B\}$为凸集\\
证明:定义仿射变换$f(x)\triangleq B-A(x)$,注意$f(x)$中的$x=[A_1,A_2,\dots,A_n]$,是多个矩阵,\\
\phantom{证明:}而$B-A(x)$返回的是一个矩阵,$f$是从高维向低维的一个映射\\
\phantom{证明:}由$S_+^n$为凸,所以$f^{-1}(S_+^n)=\{x\mid B-A(x)\succeq 0\}\Leftrightarrow\{x\mid A(x)\preceq B\}$为凸,得证

求证:椭球是球的仿射映射\\
证明:椭球:$\epsilon=\{x\mid (x-x_c)^Tp^{-1}(x-x_c)\leq 1\},\ p\in S_{++}^m$,单位球:$\{u\mid \Vert u\Vert_2\leq 1\}$\\
\phantom{证明:}定义$f(u)=p^{\frac{1}{2}}u+x_c$,则$\{f(u)\mid \Vert u\Vert_2\leq 1\}=\{p^{\frac{1}{2}}u+x_c\mid \Vert u\Vert_2\leq 1\}$\\
\phantom{证明:}令$x\triangleq p^{\frac{1}{2}}u+x_c\Rightarrow u=p^{-\frac{1}{2}}(x-x_c)$(由于$p\in S_{++}^m$,p可以求逆)\\
\phantom{证明:}回代得$\{x\mid \Vert p^{-\frac{1}{2}}(x-x_c)\Vert_2\leq 1\}=\{x\mid (x-x_c)^Tp^{-1}(x-x_c)\leq 1\}$

\textbf{透视函数(perspective function):}$P:\ R^{n+1}\to {R^n}$,$dom\ P:\ R^n\times R_{++}$\\$P(z,t)=\frac{z}{t},\ z\in R^n,\ t\in R_{++}$,凸集通过透视变换仍为凸集

考虑$R^n$内的线段,$x=(\tilde{x},x_{n+1}),\ y=(\tilde{y},y_{n+1}),\ \tilde{x},\tilde{y}\in R^n,\,x_{n+1},y_{n+1}\in R_{++},\,\theta\geq 0$,线段为$\theta x+(1-\theta) y$\\
求证:任意线段通过透视函数后仍为凸集\\
证明:P是透视函数,证明如下
\begin{align*}
	P(\theta x+(1-\theta) y)
	&=\frac{\theta \tilde{x}+(1-\theta)\tilde{y}}{\theta x_{n+1}+(1-\theta)y_{n+1}}\\
	&=\frac{\theta x_{n+1}}{\theta x_{n+1}+(1-\theta)y_{n+1}}\frac{\tilde{x}}{x_{n+1}}+\frac{(1-\theta) y_{n+1}}{\theta x_{n+1}+(1-\theta)y_{n+1}}\frac{\tilde{y}}{y_{n+1}}\\
	&=\mu P(x)+(1-\mu)P(y)
\end{align*}

求证:任意凸集的反透视映射仍为凸集,即$P^{-1}(C)=\{(x,t)\in R^{n+1}\mid \frac{x}{t}\in C,\ t>0\}$\\
证明:考虑该该反透视映射集合中的两点$(x,t)\in P^{-1}(C),\ (y,s)\in P^{-1}(C),\ 0\leq \theta\leq 1$\\
\phantom{证明:}即证:$P(\theta (x,t)+(1-\theta)(y,s))\in C$
\begin{align*}
	\text{原式}&=\frac{\theta x+(1-\theta)y}{\theta t+(1-\theta)s}\in C\\
	&=\frac{\theta t}{\theta t+(1-\theta s)}\frac{x}{t}+(1-\frac{\theta t}{\theta t+(1-\theta s)}\frac{x}{t})\frac{y}{s}\in C
\end{align*}

\textbf{线性分数函数:}仿射映射和透视映射的结合\\ [6pt]
$\delta:\ R^n\to R^{m+1}$为仿射映射,$\delta(x)=\left[\begin{array}{c}
	A \\
	C^T
\end{array}
\right]x+\left[\begin{array}{c}
	b \\
	d
\end{array}
\right]$,$A\in R^{m\times n},b\in R^m,C\in R^n,d\in R$\\ [6pt]
$P:\ R^{m+1}\to R^m$,线性分数函数$f:\ R^n\rightarrow R^m\triangleq P\circ \delta$\\ [6pt]
$f(x)=\displaystyle\frac{Ax+b}{cx+d},\ dom f=\{x\mid c^Tx+d>0\}$\\ [6pt]
经过2次保凸运算后,结果仍为凸集,即线性分数函数保凸。\\
其一定是拟凸函数,但不一定是凸函数,具体见$\alpha$-sublevel set(17/18)

求证:两个随机变量($u\in \{1,2,\dots,n\},\ v\in\{1,2,\dots,m\}$)的联合概率$\to$条件概率保凸\\
证明:$P_{ij}=P(u=i,v=j),\ f_{ij}=P(u=i\mid v=j)\Rightarrow f_{ij}=\displaystyle\frac{P_{ij}}{\sum_{k=1}^nP_{kj}}$\\
\phantom{证明:}这是一个线性分数函数,x是$[P_{1j},\dots,P_{nj}]$,分子由$[0,\dots,1,\dots,0]$点乘x,\\
\phantom{证明:}分母由1向量点乘得来,因此该映射是一个保凸的映射。



\end{document}